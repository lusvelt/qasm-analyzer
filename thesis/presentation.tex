\documentclass{beamer}
\usepackage{indentfirst}
\usepackage[italian]{babel}
\usepackage{newlfont}
\usepackage{physics}
\usepackage{amsmath}
\usepackage{amssymb}
\usepackage{amsthm}
\usepackage{xcolor}
\usepackage{listings}
\usepackage{tikz}
\usepackage{xspace}


\usetikzlibrary{quantikz}
\usetikzlibrary{shapes}
\usetikzlibrary{shapes.multipart}
\usetikzlibrary{shapes.geometric, arrows}

\tikzstyle{module} = [rectangle, minimum width=2cm, minimum height=1cm,text centered, draw=black, fill=white]
\tikzstyle{arrow} = [thick,->,>=stealth]

\usepackage[
backend=biber,
style=alphabetic,
sorting=ynt
]{biblatex}
\setbeamertemplate{bibliography item}{\insertbiblabel}

\lstset{% This applies to ALL lstlisting
  basicstyle=\small\ttfamily\color{black},%
  breaklines=true,%
  numbers=none,
  moredelim=[s][\color{green!50!black}\ttfamily]{'}{'},% single quotes in green
}%

\definecolor{MSUgreen}{RGB}{0,102,51} 
\addbibresource{bibliography.bib}

\makeatletter
\newcommand\listofframes{\@starttoc{lbf}}
\makeatother

\addtobeamertemplate{frametitle}{}{%
  \addcontentsline{lbf}{section}{\protect\makebox[2em][l]{
    \protect\usebeamercolor[fg]{structure}}
  \insertframetitle\par}
}

\usetheme{Boadilla}
\usecolortheme{spruce}
\setbeamercolor{block title}{bg=MSUgreen!10, fg=MSUgreen!60!black}
\setbeamercolor{navigation symbols dimmed}{fg=MSUgreen!40!white}
\setbeamercolor{navigation symbols}{fg=MSUgreen!40!white}
\setbeamertemplate{itemize item}{\color{MSUgreen}\textbullet}
\setbeamercolor{section in toc}{fg=MSUgreen}
\setbeamercolor{section number projected}{bg=MSUgreen,fg=white}

\title{Analisi Statica delle Risorse in QASM}
\subtitle{Stima del Numero di Qubit}
\author{Damiano Scevola}
\institute{Alma Mater Studiorum - Università di Bologna}
\date{14 luglio 2021}

\begin{document}
\abovedisplayskip=0pt
\abovedisplayshortskip=0pt
\belowdisplayskip=0pt
\belowdisplayshortskip=0pt
\begin{frame}
\titlepage
\end{frame}

\begin{frame}
    \frametitle{Sommario}
    \tableofcontents
\end{frame}
\section{Nozioni di base sulla Computazione Quantistica}

\iffalse
\begin{frame}
    \frametitle{Computazione Quantistica}
    Data una macchina (classica o quantistica), sia $n \in \mathbb{N}$ la dimensione della sua memoria (in bit o qubit).
    \begin{itemize}
        \item \textbf{Computazione classica}: la macchina in un dato momento assume un singolo stato tra i $2^n$ possibili. Le istruzioni modificano la memoria facendo evolvere tale stato in modo sequenziale.
        \item \textbf{Computazione quantistica}: uno stato quantistico è dato dalla \textit{sovrapposizione} di più stati classici detti ``di base'' (potenzialmente anche tutti i $2^n$). Una singola istruzione tiene conto di tutte le possibili interazioni fra di essi e determina lo stato successivo.
    \end{itemize}
    Passando dal classico al quantistico, si ha uno \textit{speedup} esponenziale del tempo di esecuzione: problemi prima inaffrontabili diventano così risolvibili in un tempo ragionevole dai dispositivi quantistici.
\end{frame}
\fi
\begin{frame}
    \frametitle{Qubit}
    Considerando i valori $0$ e $1$ di un bit classico si possono definire i seguenti vettori di base (usando la \textit{ket notation})
    \begin{equation*}
        \ket{0} = \begin{bmatrix} 1 \\ 0 \end{bmatrix},\ 
        \ket{1} = \begin{bmatrix} 0 \\ 1 \end{bmatrix}
    \end{equation*}
    \begin{block}{Qubit}
        Un \emph{qubit} $\ket{\psi}$ è una combinazione lineare di $\ket{0}$ e $\ket{1}$ nel campo dei numeri complessi, ovvero, dati $c_0, c_1 \in \mathbb{C}$:
        \begin{equation*}
            \ket{\psi} = c_0\ket{0}+c_1\ket{1} = c_0\begin{bmatrix} 1 \\ 0 \end{bmatrix}+ c_1\begin{bmatrix} 0 \\ 1 \end{bmatrix} = \begin{bmatrix} c_0 \\ c_1 \end{bmatrix},\ \mathit{con}\ |c_0|^2+|c_1|^2=1
        \end{equation*}
    \end{block}
    \begin{block}{Misurazione}
        Misurando $\ket{\psi}$, la probabilità di ottenere $\ket{0}$ è $|c_0|^2$, mentre la probabilità di ottenere $\ket{1}$ è $|c_1|^2$. Una volta misurato il qubit, il suo stato collassa sul valore osservato, e si perde l'informazione relativa ai valori $c_0$ e $c_1$.
    \end{block}
    
\end{frame}

\begin{frame}
    \frametitle{Registri e Entanglement}
    \begin{block}{Registri Quantistici}
        Un registro quantistico di dimensione $n$ è dato dal \textit{prodotto tensoriale} di $n$ qubit, ad esempio $\ket{\psi}=c_{00}\ket{00}+c_{01}\ket{01}+c_{10}\ket{10}+c_{11}\ket{11}$.
        Il numero massimo di stati classici sovrapponibili in un registro quantistico è $2^n$ (\textit{speedup esponenziale} rispetto alla computazione classica).
    \end{block}
    \begin{block}{Entanglement}
        Dato un registro quantistico di dimensione almeno 2, si ha un entanglement quando la misurazione di un qubit determina il valore di altri qubit senza misurarli direttamente. Esempio:
        \begin{equation*}
            \ket{\psi}=\frac{\ket{00}+\ket{11}}{\sqrt{2}}
        \end{equation*}
    \end{block}

\end{frame}

\section{Circuiti Quantistici e \texttt{OpenQASM}}

\begin{frame}
    \frametitle{Circuiti Quantistici}
    \begin{block}{Porte Logiche Quantistiche}
        Lo stato dei qubit può essere modificato applicando le porte logiche quantistiche, rappresentabili attraverso matrici unitarie.
        Porte logiche a più qubit possono introdurre entanglement.
    \end{block}
    \begin{block}{Circuiti Quantistici}
        Un circuito quantistico è una sequenza ordinata di porte logiche quantistiche, misurazioni e reset di qubit, che possono interagire con dati relativi ad una computazione classica parallela. Esempio (circuito di Deutsch-Jozsa per funzioni booleane costanti o bilanciate):
            \begin{quantikz}
            & \lstick{$\ket{\mathbf{0}}$} & \qw & \qwbundle{n}\qw & \gate{H^{\otimes n}} & \qw & \qwbundle{n}\qw & \gate[wires=2][2cm]{U_f}\gateinput{$x$}\gateoutput{$x$}& \qw & \qwbundle{n}\qw & \gate{H^{\otimes n}} & \qw & \qwbundle{n}\qw & \meter{}\\
            & \lstick{$\ket{1}$} & \qw & \qw & \gate{H} & \qw & \qw & \gateinput{$y$}\gateoutput{$y\oplus f(x)$} & \qw & \qw & \qw & \qw & \qw & \qw &
        \end{quantikz}
    \end{block}
\end{frame}

\begin{frame}[fragile]
\frametitle{Algoritmo di Deutsch-Jozsa in \texttt{OpenQASM}}
\begin{lstlisting}
OpenQASM 3.0;
include "stdgates.inc"; // Porte X e H
include "oracle.qasm"; // Oracolo U_f della funzione f
const n = 4; // Dimensione dell`input della funzione f
qreg[n] x; // Input della funzione f
qubit y; // Qubit ausiliario
reset x[0:n-1]; // |x> = |0>
creg[n] c; // Output della misurazione
reset y; // |y> = |0>
X y; // |y> = |1>
for i in [0 : n-1] {
    H x[i]; // x[i]
} // Tutti i qubit di x sono in sovrapposizione
U_f(n) x[0:n-1], y; // Applicazione di f
for i in [0 : n-1] {
    H x[i];
} // Si riporta x alla base canonica
c[0 : n-1] = measure x[0 : n-1];
\end{lstlisting}
\end{frame}

\section{Analizzatore Statico}

\begin{frame}
    \frametitle{Esecuzione Simbolica di una Subroutine}
    \begin{itemize}
        \itemsep 0em
        \item Si \textbf{inizializza} lo \emph{store} assegnando ai parametri formali valori simbolici.
        \item I valori simbolici compaiono nelle \textbf{espressioni} come letterali, ad esempio \texttt{1+2*(\$0-\$1)}, dove \texttt{\$0} e \texttt{\$1} sono simboli.
        \item Per le \textbf{dichiarazioni} di variabili classiche si prepara un \emph{item} nello \emph{store} contenente metadati sul tipo e sulla dimensione della variabile.
        \item Per gli \textbf{assegnamenti} si valuta l'espressione \emph{simbolica} di destra e la si inserisce nell'\emph{item} dello \emph{store} corrispondente alla variabile di sinistra.
        \item Per le \textbf{misurazioni} di qubit si segue la procedura dell'assegnamento classico, dove però il valore assegnato è un simbolo fresco.
        \item Per le istruzioni di \textbf{selezione} (\texttt{if-then-else}), si invoca l'SMT solver per verificare se la condizione è soddisfacibile, poi si crea un ramo di esecuzione per ogni porzione di codice da simulare, tenendo traccia delle condizioni sui simboli che valgono in ognuna.
        \item Per i \textbf{cicli definiti} (\texttt{for}), si aggiungono sullo \emph{stack di esecuzione} $n$ blocchi, ognuno contenente le istruzioni del corpo del ciclo, separati da meta-istruzioni per modificare il valore dell'iteratore nello store.
    \end{itemize}
\end{frame}

\begin{frame}
    \frametitle{Stima del Numero di Qubit durante l'Esecuzione Simbolica}
    \begin{itemize}
        \item Si utilizzano un array \emph{potentialEntanglements} e un insieme \emph{actualQubits}, entrambi \textbf{inizialmente vuoti}.
        \item Quando si simula l'applicazione di una \textbf{porta logica quantistica} $G$ a più qubit, si crea un insieme $X$ che contiene tutti i qubit coinvolti in $G$. $X$ viene inserito in \emph{potentialEntanglements} e fuso con tutti gli altri insiemi in cui compare almeno un qubit già presente in $X$.
        \item Gli insiemi presenti in \emph{potentialEntanglements} sono \textbf{disgiunti}.
        \item Quando un qubit $q$ viene \textbf{misurato}, si individua, se presente, l'insieme $\bar{X}$ contenente $q$ in \emph{potentialEntanglements}. $\bar{X}$ viene rimosso dall'array e tutti i qubit in esso presenti vengono inseriti nell'insieme \emph{actualQubits}.
        \item Il \textbf{numero di qubit} massimo stimato è pari alla cardinalità di \emph{actualQubits}.
    \end{itemize}
\end{frame}

\begin{frame}
    \frametitle{Architettura dell'Analizzatore}
    \begin{tikzpicture}[node distance=3cm]
        \node (parser) [module] {Parser};
        \node (subcl) [module, below of=parser, text width=2cm] {Subroutine Classifier};
        \node (see) [module, right=4cm of parser, text width=2cm] {Symbolic Execution Engine};
        \node (smt) [module, above of=see] {SMT Solver};
        \node (expr) [module, below of=see, text width=2cm] {Expression Manipulator};

        \path[<-] (parser.north) edge[text width=2cm] node[right]{Programma \texttt{OpenQASM}} ++(0em,4em);
        \path[->] (parser.south) edge[text width=2cm] node[right]{Albero di Parsing} (subcl.north);
        \draw[->] (subcl.east) -- ++(3em, 0em) |- ++(0em, 7.8em) -- node[below]{Subroutine} (see.west);
        \path[->] (see.east) edge[text width=3cm] node[xshift=1em]{Symbolic Execution Tree} ++(8em, 0em);
        \path[->] (see.north) edge[text width=2cm, xshift=-1em] node[xshift=-2.2em]{Espressione Simbolica} (smt.233);
        \path[<-] (see.north) edge[text width=2cm, xshift=1em] node[xshift=3em]{Soddisfacibile?} (smt.307);
        \path[->] (see.south) edge[text width=2cm, xshift=-1em] node[xshift=-1.4em]{Nodo di Parsing, Contesto} (expr.125);
        \path[<-] (see.south) edge[text width=2cm, xshift=1em] node[xshift=3em]{Espressione Simbolica} (expr.55);
    \end{tikzpicture}
\end{frame}

\section{Conclusioni}
\begin{frame}
    \frametitle{Conclusioni}
    \begin{itemize}
        \item Implementare fisicamente i qubit è altamente costoso, quindi i dispositivi quantistici attualmente esistenti ne hanno un numero limitato: l'analizzatore presentato si propone come strumento per ottimizzare le risorse disponibili.
        \item La stima del numero di qubit \textit{effettivamente} utilizzati viene effettuata contando i qubit che sono (potenzialmente) \emph{entangled} con qubit misurati, o sono misurati essi stessi.
        \item Quella presentata è una prima versione dell'analizzatore: le limitazioni e le potenzialità di sviluppo sono numerose.
    \end{itemize}
\end{frame}


%\begin{frame}
%    \frametitle{Bibliografia}
%    \printbibliography
%\end{frame}

\end{document}

% Sommario
% 1) Computazione quantistica: stati quantici, qubit e speedup
% 2) Computazione quantistica: gates, entanglement e measurement
% 3) Problema di Deutsch-Jozsa e Circuito (anche in QASM)
% 4) OpenQASM: particolarità del linguaggio
% 5) Esecuzione simbolica di una subroutine: esempio
% 6) Stima del numero di qubit: esempio con entanglement e measurement
% 7) Implementazione dell'analizzatore: moduli e diagramma di flusso
% Conclusioni