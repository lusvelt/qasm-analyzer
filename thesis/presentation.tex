\documentclass{beamer}
\usepackage{indentfirst}
\usepackage[italian]{babel}
\usepackage{newlfont}
\usepackage{physics}
\usepackage{amsmath}
\usepackage{amssymb}
\usepackage{amsthm}
\usepackage{xcolor}
\usepackage{listings}
\usepackage{tikz}
\usepackage{xspace}
\usepackage[
backend=biber,
style=alphabetic,
sorting=ynt
]{biblatex}
\setbeamertemplate{bibliography item}{\insertbiblabel}

\definecolor{MSUgreen}{RGB}{0,102,51} 
\addbibresource{bibliography.bib}

\makeatletter
\newcommand\listofframes{\@starttoc{lbf}}
\makeatother

\addtobeamertemplate{frametitle}{}{%
  \addcontentsline{lbf}{section}{\protect\makebox[2em][l]{
    \protect\usebeamercolor[fg]{structure}}
  \insertframetitle\par}
}

\usetheme{Boadilla}
\usecolortheme{spruce}
\setbeamercolor{block title}{bg=MSUgreen!10, fg=MSUgreen!60!black}
\setbeamercolor{navigation symbols dimmed}{fg=MSUgreen!40!white}
\setbeamercolor{navigation symbols}{fg=MSUgreen!40!white}
\setbeamertemplate{itemize item}{\color{MSUgreen}\textbullet}

\title{Analisi Statica delle Risorse in QASM}
\subtitle{Stima del Numero di Qubit}
\author{Damiano Scevola}
\institute{Alma Mater Studiorum - Università di Bologna}
\date{14 luglio 2021}

\begin{document}
\abovedisplayskip=0pt
\abovedisplayshortskip=0pt
\belowdisplayskip=0pt
\belowdisplayshortskip=0pt
\begin{frame}
\titlepage
\end{frame}

\begin{frame}
    \frametitle{Sommario}
    \listofframes
\end{frame}

\begin{frame}
    \frametitle{Computazione Quantistica}
    Data una macchina (classica o quantistica), sia $n \in \mathbb{N}$ la dimensione della sua memoria (in bit o qubit).
    \begin{itemize}
        \item \textbf{Computazione classica}: la macchina in un dato momento assume un singolo stato tra i $2^n$ possibili. Le istruzioni modificano la memoria facendo evolvere tale stato in modo sequenziale.
        \item \textbf{Computazione quantistica}: uno stato quantistico è dato dalla \textit{sovrapposizione} di più stati classici detti ``di base'' (potenzialmente anche tutti i $2^n$). Una singola istruzione tiene conto di tutte le possibili interazioni fra di essi e determina lo stato successivo.
    \end{itemize}
    Passando dal classico al quantistico, si ha uno \textit{speedup} esponenziale del tempo di esecuzione: problemi prima inaffrontabili diventano così risolvibili in un tempo ragionevole dai dispositivi quantistici.
\end{frame}

\begin{frame}
    \frametitle{Qubit e Registri Quantistici}
    Prendendo come stati di base i valori $0$ e $1$ di un bit classico si possono definire i seguenti vettori (usando la \textit{ket notation})
    \begin{equation*}
        \ket{0} = \begin{bmatrix} 1 \\ 0 \end{bmatrix},\ 
        \ket{1} = \begin{bmatrix} 0 \\ 1 \end{bmatrix}
    \end{equation*}
    \begin{block}{Qubit}
        Un \emph{qubit} $\ket{\psi}$ è una combinazione lineare di $\ket{0}$ e $\ket{1}$ nel campo dei numeri complessi, ovvero (dati $c_0, c_1 \in \mathbb{C}$):
        \begin{equation*}
            \ket{\psi} = c_0\ket{0}+c_1\ket{1} = c_0\begin{bmatrix} 1 \\ 0 \end{bmatrix}+ c_1\begin{bmatrix} 0 \\ 1 \end{bmatrix} = \begin{bmatrix} c_0 \\ c_1 \end{bmatrix},\ \mathit{con}\ |c_0|^2+|c_1|^2=1
        \end{equation*}
    \end{block}
    \begin{block}{Registri Quantistici}
        Un registro quantistico di dimensione $n$ è dato dal \textit{prodotto tensoriale} di $n$ qubit, ad esempio $\ket{\psi}=c_{00}\ket{00}+c_{01}\ket{01}+c_{10}\ket{10}+c_{11}\ket{11}$
    \end{block}
\end{frame}

\begin{frame}
    \frametitle{Misurazioni e Entanglement}
    \begin{block}{Misurazione}
        Dato un qubit $\ket{\psi}=\begin{bmatrix}c_0\\c_1\end{bmatrix}$, il risultato di una sua misurazione può essere $\ket{0}$ con probabilità $|c_0|^2$, oppure $\ket{1}$ con probabilità $|c_1|^2$. Una volta misurato il qubit, il suo stato collassa sul valore osservato, e si perde l'informazione relativa allo stato precedente.
    \end{block}
    \begin{block}{Entanglement}
        Dato un registro quantistico di dimensione almeno 2, si ha un entanglement quando la misurazione di un qubit determina il valore di altri qubit senza misurarli direttamente. Esempio:
        \begin{equation*}
            \ket{\psi}=\frac{\ket{00}+\ket{11}}{\sqrt{2}}
        \end{equation*}
    \end{block}

\end{frame}

\begin{frame}
    \frametitle{Circuiti Quantistici}
    \begin{block}{Porte Logiche Quantistiche}
        Lo stato dei qubit può essere modificato applicandogli le porte logiche quantistiche, rappresentabili matematicamente attraverso matrici unitarie. Esempi:
        \begin{equation*}
            H=\begin{bmatrix} \frac{1}{\sqrt{2}} & \frac{1}{\sqrt{2}}\\
                \frac{1}{\sqrt{2}} & -\frac{1}{\sqrt{2}}\end{bmatrix},\ ^CX=\begin{bmatrix}1&0&0&0\\0&1&0&0\\0&0&0&1\\0&0&1&0\end{bmatrix}
        \end{equation*}
    \end{block}
\end{frame}

\begin{frame}
    \frametitle{Esecuzione Simbolica}
\end{frame}

\begin{frame}
    \frametitle{Stima del Numero di Qubit}
\end{frame}

\begin{frame}
    \frametitle{Implementazione dell'Analizzatore}
\end{frame}

\begin{frame}
    \frametitle{Conclusioni}
\end{frame}

\begin{frame}
    \frametitle{Bibliografia}
    \printbibliography
\end{frame}
\end{document}

% Sommario
% 1) Computazione quantistica: stati quantici, qubit e speedup
% 2) Computazione quantistica: gates, entanglement e measurement
% 3) Problema di Deutsch-Jozsa e Circuito (anche in QASM)
% 4) OpenQASM: particolarità del linguaggio
% 5) Esecuzione simbolica di una subroutine: esempio
% 6) Stima del numero di qubit: esempio con entanglement e measurement
% 7) Implementazione dell'analizzatore: moduli e diagramma di flusso
% Conclusioni